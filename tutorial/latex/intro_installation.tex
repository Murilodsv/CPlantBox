This installation guideline is for CPlantBox on Linux systems (e.g. Ubuntu). 
If on a recent Ubuntu system, the C++ compiler and python that come with the distribution are recent enough. Otherwise, please make sure you have a recent C++ compiler (e.g.   ), fortran compiler (sudo apt-get install gfortran) and python3. \\

If you wish to be able to follow the examples provided in the Jupyter Notebooks, it is recommended to install the Anaconda Python distribution: 
\begin{itemize}
\item Install curl if not yet available: \lstinline{sudo apt-get install curl}
\item Download and install Anaconda (adapt the version accordingly):\\
\lstinline{cd /tmp}\\
\lstinline{curl https://repo.anaconda.com/archive/Anaconda3-2021.11-Linux-x86_64.sh --output anaconda.sh} \\
\lstinline{bash anaconda.sh} \\
\lstinline{source ~/.bashrc}
\end{itemize}

CPlantBox can be installed using either the install script installCPlantBox.py for CPLantBox only, or installDumuxRosi\_Ubuntu.py for CPlantBox and dumux-rosi. First make sure Git ist installed, clone the CPlantBox repository, and the run the install script:
\begin{itemize}
 \item Install Git: \lstinline{sudo apt-get install git}
 \item Go to your base folder and clone repository: \lstinline{git clone https://github.com/Plant-Root-Soil-Interactions-Modelling/CPlantBox.git}
 \item Run install script: \lstinline{python3 CPlantBox/installDumuxRosi_Ubuntu.py}  
\end{itemize}

% - Install git: \\
% \\
% - Install cmake:\\
% \lstinline{sudo apt-get install cmake}\\
% - Install libboost:\\
% \lstinline{sudo apt-get install libboost-all-dev}\\
% - Install pip:\\
% \lstinline{sudo apt-get install python3-pip}\\
% - Install the python package numpy:\\
% \lstinline{pip3 install numpy}\\
% - Install the python package scipy:\\
% \lstinline{pip3 install scipy}\\
% - Install the python package matplotlib:\\
% \lstinline{pip3 install matplotlib}\footnote{Known bug in ubuntu 18.04: needs sudo apt-get install libfreetype6-dev libxft-dev installed before.}\\
% - Install the python package VTK:\\
% \lstinline {pip3 install vtk}\\
% - Install the java runtime environment:\\
% \lstinline{sudo apt-get install default-jre}\\
% - Install Paraview\\
% \lstinline{sudo apt-get install paraview}\\
% To build CPlantBox and its python shared library, move again into the CPlantBox folder and type into the console:\\
% \lstinline {cd CPlantBox}\\
% \lstinline{cmake .}\footnote{It may be necessary on your installation to check the CPlantBox/src/CMakeLists.txt file regarding required python version and out-commenting line 34.}\\
% \lstinline{make}\\
% (If building CPlantBox on the IBG-3 cluster, two lines in the file \lstinline{CPlantBox/CMakeLists.txt} need to be outcommented before:\\ 
% \lstinline{set(CMAKE_C_COMPILER "/usr/bin/gcc")}\\
% \lstinline{set(CMAKE_CXX_COMPILER "/usr/bin/g++")})\\
% 
After successfully compiling CPlantBox the Python library \emph{plantbox} should be available on your system. 
