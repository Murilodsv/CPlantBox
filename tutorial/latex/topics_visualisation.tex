

\subsubsection*{How to make an animation} \label{ssec:animation}

In order to create an animation in Paraview we have to consider some details. The main idea is to export the result file as segments using the class SegmentAnalyser. A specific frame is then obtained by thresholding within Paraview using the segments creation times. In this way we have to only export one vtp file. 

We modify example1b.py to demonstrate how to create an animation.

\lstinputlisting[language=Python, caption=Example 4c (modified from Example 1b)]{examples/example3e_animation.py} 

\begin{itemize}

\item[11,12] Its important to use a small resolution in order to obtain a smooth animation. L18 set the axial resolution to 0.1 cm. 

\item[19,29] Instead of saving the root system as polylines, we use the SegmentAnalyser to save the root system as segments.

\item[22,23] It is also possible to make the root system periodic in the visualization in $x$ and $y$ direction to mimic field conditions.

\item[26-28] We save the geometry as Python script for the visualization in ParaView.

\end{itemize}

After running the script we perform the following operations Paraview to create the animation:
\begin{enumerate}
 \item Open the .vtp file in ParaView (File$\rightarrow$Open...), and open tutorial/examples/python/results/example\_3e.vtp.
 \item Optionally, create a tube plot with the help of the script tutorial/pyscript/rsTubePlot.py (Tools$\rightarrow$Python Shell, press 'Run script').
 \item Run the script tutorial/pyscript/rsAnimate.py (Tools$\rightarrow$Python Shell, press 'Run script'). The script creates the threshold filter and the animation. 
 \item Optionally, visualize the domain boundaries by running the script tutorial/examples/python/results/example\_4e.py (Tools$\rightarrow$Python Shell, press 'Run script'). Run after the animation script (otherwise it does not work).  
 \item Use File$\rightarrow$Save Animation... to render and save the animation. Pick quality ($<$100 \%), and the frame rate in order to achieve an appropriate video length, e.g. 300 frames with 50 fps equals 6 seconds. The resulting files might be uncompressed and are very big. The file needs compression, for Linux e.g. ffmpeg -i in.avi -vcodec libx264 -b 4000k -an out.avi, produces high quality and tiny files, and it plays with VLC.
\end{enumerate}


