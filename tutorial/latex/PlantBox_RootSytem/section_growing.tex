\newpage
\section{Coupling growing rootsystems to numerical models} \label{sec:growing}


In this section it is described how information about the last time step can be retrieved, 
and how we can incrementally obtain the root system from only new nodes and segments. 
These methods are especially important if we couple to other numerical software like DuMux


In the previous section root responses were described in a static soil. 
In this section we will extend this to a dynamic soil setting, where we update the soil in the simulation loop, and then update the root system iteratively for small time steps. 

General properties of the soil, are passed to the root model via a look up method SoilLookUp::getValue in the class SoilLookUp. 
In the following subsection we will first describe this metod and some implemented usefull extensions of the SoilLookUp class (Section \ref{sec:soil}), 
and show how we can create an interface to a generic soil in Pyhton (Section \ref{sec:usersoil}). 

\subsection{Mapping of growing roots and underlying soil} 


\lstinputlisting[firstline=1, language=Python, caption=Example 7a]{../../examples/python/example7a_mapping.py}


\subsection{Coupling a dynamic root system to DuMux} \label{sec:dumux_dyn_coupling}

Very little to do (simulate call to the root system) \\

Feedback: Make root growth dependent on soil (hydrotropims vs. no hydrotropism

\section{Todos}
Topics that are not covered yet or should be improved

\begin{itemize}

\item The DuMux binding and MPI

\item Elongation rate according to Moacir et al.

\item Delay based, versus length based lateral emergance.

\item Better branching probability example.

\item Periodicity without DuMux coupling, but with a soil 3d grid (untested, and an example is missing)

\item Carbon limited root system growth

\end{itemize}

